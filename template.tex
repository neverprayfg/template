\documentclass[UTF8, a4paper, titlepage, twoside]{ctexart}
\usepackage{geometry}
\usepackage{hyperref}
\usepackage{graphicx}
\usepackage{titlesec}
\usepackage{minted}
\usepackage[dvipsnames, svgnames, x11names]{xcolor}
% 设置边距和页面布局
\geometry{left=2.5cm, right=1.5cm, top=1.5cm, bottom=1.5cm, inner=2cm, outer=1cm}
% ------------------------------子章节设置------------------------------
\setcounter{secnumdepth}{4}
\setcounter{tocdepth}{4}
\AtBeginDocument{
	\setlength{\abovedisplayskip}{0.2em}
	\setlength{\belowdisplayskip}{0.2em}
	\setlength{\parskip}{0.5em}
}
\title{	
	\normalfont\normalsize
	\textsc{Beijing Normal University}\\ 
	\vspace{25pt} 
	\rule{\linewidth}{0.5pt}\\ 
	\vspace{20pt} 
	{\Huge Template}\\ 
	\vspace{12pt} 
	\rule{\linewidth}{2pt}\\ 
	\vspace{12pt} 
}
\author{Pray Never} 
\date{\normalsize\today} 
\begin{document}
\maketitle
\tableofcontents
\newpage
\section{geometry}
\subsection{convex.cpp}
\begin{minted}[frame=lines, fontsize=\small, linenos]{C++}
#ifdef IGNORE_THIS_FILE
  struct Point {
     ll x,y;
  };
  auto andrew = [](vector<Point>& p) -> vector<Point> {    // 传入下标从零开始的点数组,返回凸包数组
    auto cmp = [](Point &a, Point &b) -> bool {
      if(a.x != b.x) return a.x < b.x;
      return a.y < b.y;
    };
    auto cross = [](Point &u, Point &v, Point &w) -> bool {
      ll x1 = u.x - v.x, y1 = u.y - v.y;
      ll x2 = w.x - v.x, y2 = w.y - v.y;
      return x1 * y2 - x2 * y1 > 0; //如果不希望在凸包的边上有输入点。把 > 改成 >=
    };
    sort(p.begin(), p.end(), cmp);
    int n = p.size(), m = 0;
    vector<Point> res(n + 1);
    for(int i = 0; i < n; ++i){
      while(m > 1 && !cross(res[m - 1],res[m - 2], p[i])) --m;
      res[m++] = p[i];
    }
    int kk = m;
    for(int i = n - 2; i >= 0; i--){
      while(m > kk && !cross(res[m - 1], res[m - 2], p[i])) --m;
      res[m++] = p[i];
    }
    if(n > 1) --m;//凸包有m个顶点
    res.erase(res.begin() + m, res.end());
    return res;
  };
#endif
\end{minted}

\clearpage
\section{graph}
\subsection{LCA.cpp}
\begin{minted}[frame=lines, fontsize=\small, linenos]{C++}
#ifdef IGNORE_THIS_FILE
  vector<vector<int> > a(n + 1, vector<int>(20)), v(n + 1);
  vector<int> dep(n + 1);
  auto build = [&](int u, int fa, auto&& self) -> void {
    dep[u] = dep[fa] + 1, a[u][0] = fa;
    for(int i = 1; i <= 19; ++i)
      a[u][i] = a[a[u][i - 1]][i - 1];
    for(int i: v[u]){
      if(i == fa) continue;
      self(i, u, self);
    }
  };
  auto lca = [&](int x, int y) -> int {
    if(dep[y] > dep[x]) swap(x, y);
    for(int i = 19; i >= 0; --i){
      if(dep[a[x][i]] >= dep[y]) 
        x = a[x][i];
    }
    if(x == y) return x;
    for(int i = 19; i >= 0; --i){
      if(a[x][i] != a[y][i])
        x = a[x][i], y = a[y][i];
    }
    return a[x][0];
  };
#endif



\end{minted}

\subsection{tarjan.cpp}
\begin{minted}[frame=lines, fontsize=\small, linenos]{C++}
#ifdef IGNORE_THIS_FILE
  int dfn_cnt = 0, scc_cnt = 0;
  vector<int> dfn(n + 1), low(n + 1), scc_id(n + 1); 
  stack<int> s;
  vector<bool> in_stack(n + 1);
  vector<vector<int> > v(n + 1), scc(n + 1);
  // scc_id是每个节点所属于的scc编号, scc是这个编号下的所有节点
  auto tarjan = [&](int u, auto&& self) -> void {
    dfn[u] = low[u] = ++dfn_cnt;
    s.push(u), in_stack[u] = true;
    for(int i: v[u]){
      if(!dfn[i]){
        self(i, self);
        low[u] = min(low[u], low[i]);
      }
      else if(in_stack[i])
        low[u] = min(low[u], dfn[i]);
    }
    if(dfn[u] == low[u]){
      int tp;
      ++scc_cnt;
      do{
        tp = s.top();
        scc_id[tp] = scc_cnt;
        scc[scc_cnt].push_back(tp);
        in_stack[tp] = false;
        s.pop();
      } while(tp != u);
    }
  };
#endif
\end{minted}

\clearpage
\section{heading}
\subsection{debug.h}
\begin{minted}[frame=lines, fontsize=\small, linenos]{C++}
 #include <bits/stdc++.h>
 #define typet typename T
 #define typeu typename U
 #define types typename... Ts
 #define tempt template <typet>
 #define tempu template <typeu>
 #define temps template <types>
 #define tandu template <typet, typeu>

 tandu std::ostream& operator<<(std::ostream& os, const std::pair<T, U>& p) {
 return os << '<' << p.ff << ',' << p.ss << '>';
 } 
 template <
 typet, typename = decltype(std::begin(std::declval<T>())),
 typename = std::enable_if_t<!std::is_same_v<T, std::string>>>
 std::ostream& operator<<(std::ostream& os, const T& c) {
 auto it = std::begin(c);
 if (it == std::end(c)) return os << "{}";
 for (os << '{' << *it; ++it != std::end(c); os << ',' << *it);
 return os << '}';
 }
 #define debug(arg...) \
 do { \
  std::cerr << "[" #arg "] :";
  dbg(arg); \
 }while(false)

 temps void dbg(Ts... args) {
 (..., (std::cerr << ' ' << args));
 std::cerr << '\n';
 }

\end{minted}

\subsection{duipai.cpp}
\begin{minted}[frame=lines, fontsize=\small, linenos]{C++}
#ifdef IGNORE_THIS_FILE
  system("g++ -std=c++2a wa.cpp -o/wa");
  system("g++ -std=c++2a ac.cpp -o/ac");
  system("g++ -std=c++2a gen.cpp -o/gen");
  for(int i = 1; i <= 50; i++){
    std::cerr << "Test" << i << " : ";
    system("./gen > gen.in");
    system("./ac < gen.in > ac.out");
    system("./wa < gen.in > wa.out");
    if (system("diff ac.out wa.out")) {
      std::cerr << "ERR\n";
      return 0;
    }
    std::cerr << "AC\n";
  }
#endif  
\end{minted}

\subsection{heading.cpp}
\begin{minted}[frame=lines, fontsize=\small, linenos]{C++}
  #include <bits/stdc++.h>
  using namespace std;
  using ll = long long;
  using i128 = __int128;
  #define ff first
  #define ss second
  #include "debug.h"
  constexpr int mod = 998244353;
  constexpr ll INF = 1e18;
  constexpr double pi = 3.141592653589793;
  constexpr double eps = 1e-6;


\end{minted}

\clearpage

\end{document}
